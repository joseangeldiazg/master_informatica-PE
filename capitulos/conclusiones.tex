\chapter{Conclusiones}

Para finalizar, podemos concluir que el trabajo realizado en la OficinaWeb no solo cubre las necesidades de formaci�n, sino que las supera, sobre los 6ECTS estipulados de la asignatura Pr�cticas de Empresa. Para que esto quede aun mejor constatado, podemos trazar a modo de resumen un peque�o paralelismo entre asignaturas y competencias del master y tareas realizadas en la \textbf{OficinaWeb}.

\begin{itemize}
\item El uso de Trello para gesti�n interna del equipo de trabajo y comunicaciones, es algo que est� �ntimamente ligado a la asignatura Planificaci�n y Gesti�n Proyectos Inform�ticos. En el d�a a d�a de nuestro trabajo lidiamos con prioridades, checklists, contratos, usuarios y clientes? algo que ha sido ampliamente estudiado en esta asignatura. 
\item Dise�o de webs teniendo en cuenta la accesibildiad y usabilidad, este campo es estudiado en la asignatura Dise�o de Interfaces de Usuario Interacctivas. 
\item Desarrollo web, en asignaturas como Sistemas Software Basados en Web, hemos visto frameworks y librer�as que hemos extendido al trabajo como Bootstrap para el CSS, ademas de la programaci�n web en si, obviamente ligada a esta asignatura. 
\item Requisitos. En asignaturas como Desarrollo de Software Basado en Compontenes, se estudian los requisitos, algo que usamos casi semanalmente en nuestro trabajo. 
\item Bases de datos, han sido estudiadas en asignaturas como, Cloud Computing: Servicios y Aplicaciones y como hemos visto en anteriores secciones el conocimiento de las bases de datos ha sido extrapolado a nuestro trabajo. 
\end{itemize}

Cabe destacar, que la anterior lista hace referencia a las asignaturas del master y no a las del grado, si hubi�ramos tenido estas en cuenta, la lista ser�a mucho mayor, 